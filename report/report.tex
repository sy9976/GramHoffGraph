\documentclass[A_4paper,12pt]{article}
\usepackage{polski}
%\usepackage[version=3]{mhchem} % Package for chemical equation typesetting
%\usepackage{siunitx} % Provides the \SI{}{} and \si{} command for typesetting SI units
\usepackage{graphicx, xcolor} % Required for the inclusion of images
\usepackage{natbib} % Required to change bibliography style to APA
\usepackage{amsmath} % Required for some math elements 
\usepackage[utf8]{inputenc}
\setlength\parindent{0pt} % Removes all indentation from paragraphs
\usepackage{svg}
\usepackage{geometry}
\usepackage{afterpage}
\usepackage{caption}
\usepackage{layout}
\usepackage{tabularx}
\usepackage{wrapfig}
\usepackage{float}
\usepackage{capt-of}

\renewcommand{\labelenumi}{\alph{enumi}.} % Make numbering in the enumerate environment by 
%\usepackage{times} % Uncomment to use the Times New Roman font

\title{Laboratorium Informatyki w Medycynie \\ 1 punkt kontrolny} % Title
\author{Szymon \textsc{Gramza} 109785  \\ Przemysław \textsc{Hoffmann} 109786} % Author name

\date{08.04.2015r.} % Date for the report

\begin{document}

\maketitle % Insert the title, author and date

\begin{center}
\begin{tabular}{l r}
Prowadzący: & dr inż Tomasz Pawlak \\
Temat zadania: & tomograf komputerowy
\end{tabular}
\end{center}

\newpage
 kilka słów o zadaniu, proponowany opis rozwiązania postawionego zadania, 
 proponowana architektura aplikacji, technologie, 
 spodziewane problemy i zarys rozwiązania, możliwe rozszerzenia.
\section{Opis problemu}
Tematem projektu zaliczeniowego jest symulator 

\section{Technologia}
\section{Metoda rozwiązania}
\section{Plan dalszej pracy}


\bibliographystyle{apalike}
\bibliography{sample}

\end{document}