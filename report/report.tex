\documentclass[A_4paper,12pt]{article}
\usepackage{polski}
%\usepackage[version=3]{mhchem} % Package for chemical equation typesetting
%\usepackage{siunitx} % Provides the \SI{}{} and \si{} command for typesetting SI units
\usepackage{graphicx, xcolor} % Required for the inclusion of images
\usepackage{natbib} % Required to change bibliography style to APA
\usepackage{amsmath} % Required for some math elements 
\usepackage[utf8]{inputenc}
\setlength\parindent{0pt} % Removes all indentation from paragraphs
\usepackage{svg}
\usepackage{geometry}
\usepackage{afterpage}
\usepackage{caption}
\usepackage{layout}
\usepackage{tabularx}
\usepackage{wrapfig}
\usepackage{float}
\usepackage{capt-of}

\renewcommand{\labelenumi}{\alph{enumi}.} % Make numbering in the enumerate environment by 
%\usepackage{times} % Uncomment to use the Times New Roman font

\title{Laboratorium Informatyki w Medycynie \\ 1 punkt kontrolny} % Title
\author{Szymon \textsc{Gramza} 109785  \\ Przemysław \textsc{Hoffmann} 109786} % Author name

\date{08.04.2015r.} % Date for the report

\begin{document}

\maketitle % Insert the title, author and date

\begin{center}
\begin{tabular}{l r}
Prowadzący: & dr inż Tomasz Pawlak \\
Temat zadania: & tomograf komputerowy
\end{tabular}
\end{center}

\newpage
 kilka słów o zadaniu, proponowany opis rozwiązania postawionego zadania, 
 proponowana architektura aplikacji, technologie, 
 spodziewane problemy i zarys rozwiązania, możliwe rozszerzenia.
 
\section{Opis problemu}
Tematem projektu zaliczeniowego jest symulator tomografu komputerowego.
Według wymagań symulator ten powinien pozwalać na:
\begin{itemize}
\item akwizycję rzutów obrazów 1D z zadanego obrazu 2D
\item prezentacje użytkownikowi tych rzutów
\item rekonstrukcję obrazów 2D z rzutów 1D przy użyciu odwrotnej transformaty Radona
\item prezentację zrekonstruowanego obrazu
\end{itemize}

\subsection{Zasada działania}
Zasada działanie tomografu opiera się na pochłanianiu promieniowania rentgenowskiego przez ludzkie narządy.
Niestety, narządy organizmu ludzkiego wzajemnie się przysłaniają, co prowadzi do nakładania się na siebie obrazów poszczególnych struktur wewnętrznych człowieka.
Zauważono, że wykonanie większej (niż jeden) liczby zdjęć radiologicznych z różnych pozycji lampy i detektora względem badanego obiektu, a następnie obejrzenie zdjęć w stroboskopie prowadzi do poprawienia jakośi obrazu. Wprowadzono więc ruch lampy rentgenowskiej detektora wzgledem obiektu.

\subsubsection{Generacje tomografów}
Od czasów wynalezienia tomografu zmieniały koncepcje i usprawnienia mające na celu zwiększenie ich szybkości i wydajności.
W związku z tym powastała poniższa klasyfikacja na generacje:
\begin{itemize}
\item Generacja I - skaner składał się z pojedynczego emitera i detektora. Lampa i detektor wykonywały ruchy translacyjne i rotacyjne.
\item Generacja II - zwiekszono liczbę detektorów co zmniejszyło liczbę ruchów translacyjnych lampy.
\item Generacja III - wyeliminowano ruch translacyjny poprzez rozmieszczenie detektorów na łuku pierścienia obracającego się razem z lampą dookoła pacjenta.
\item Generacja IV - detektory umieszczone zostały na stałe na pierścienu, ruch obrotowy wykonuje tylko lampa.
\end{itemize}

W projekcie rozważa się implementację czwartej generacji.

\subsubsection{Transformata Radona}
Używana do dokonania transformacji wielu projekcji 1D w jeden obraz 2D.
W roku 1905 W. Radon udowodnił następujące twierdzenie: „Obraz obiektu dwuwymiarowego można zrekonstruować na podstawie nieskończone ilości rzutów jednowymiarowych”. Rzutowanie to odpowiada wykonywaniu na obiekcie pewnej transformacji, nazywanej Transformacją Radona. 
Po dokonaniu transformacji z otrzymanych wyników otrzymyje się sinogram, będący wizualizacją
Dokonanie na wynikach rzutowania Odwrotnej Transformacji Radona umożliwia zrekonstruowanie obrazu obiektu.

\subsubsection{Algorytm Brens}


\section{Technologia}
Program zaimplementowano w języku Python korzystając z następujących biibliotek:
\begin{itemize}
\item skimage - wczytanie obrazu
\end{itemize}

\section{Metoda rozwiązania}
[transformata]
[odwrotna]
[implementacja brasennndsfjds]

\section{Plan dalszej pracy}
[zrobimy reszte]


\bibliographystyle{apalike}
\bibliography{sample}

\end{document}