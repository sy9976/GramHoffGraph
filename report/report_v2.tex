\documentclass[A_4paper,12pt]{article}
\usepackage{polski}
%\usepackage[version=3]{mhchem} % Package for chemical equation typesetting
%\usepackage{siunitx} % Provides the \SI{}{} and \si{} command for typesetting SI units
\usepackage{graphicx, xcolor} % Required for the inclusion of images
\usepackage{natbib} % Required to change bibliography style to APA
\usepackage{amsmath} % Required for some math elements 
\usepackage[utf8]{inputenc}
\setlength\parindent{0pt} % Removes all indentation from paragraphs
\usepackage{svg}
\usepackage{geometry}
\usepackage{afterpage}
\usepackage{caption}
\usepackage{layout}
\usepackage{tabularx}
\usepackage{wrapfig}
\usepackage{float}
\usepackage{capt-of}

\renewcommand{\labelenumi}{\alph{enumi}.} % Make numbering in the enumerate environment by 
%\usepackage{times} % Uncomment to use the Times New Roman font

\title{Laboratorium Informatyki w Medycynie \\ 3 punkt kontrolny} % Title
\author{Szymon \textsc{Gramza} 109785  \\ Przemysław \textsc{Hoffmann} 109786} % Author name

\date{22.05.2015r.} % Date for the report

\begin{document}

\maketitle % Insert the title, author and date

\begin{center}
\begin{tabular}{l r}
Prowadzący: & mgr inż. Tomasz Pawlak \\
Temat zadania: & symulator tomografu komputerowego
\end{tabular}
\end{center}

\newpage

\section{Opis problemu}
Tematem projektu zaliczeniowego jest symulator tomografu komputerowego.
Według wymagań symulator ten powinien pozwalać na:
\begin{itemize}
\item akwizycję rzutów obrazów 1D z zadanego obrazu 2D
\item prezentacje użytkownikowi tych rzutów
\item rekonstrukcję obrazów 2D z rzutów 1D przy użyciu odwrotnej transformaty Radona
\item prezentację zrekonstruowanego obrazu
\end{itemize}

\subsection{Zasada działania}
Tomograf w ogólności składa się z lampy rentgenowskiej, będącej emiterem wiązki promieniowania rentegnowskiego, stołu na którym leży pacjent(badany obiekt) oraz  detektora znajdującego się po przeciwległej stronie stołu względem emitera.
Zasada działanie tomografu opiera się na pochłanianiu promieniowania rentgenowskiego przez ludzkie narządy.
Niestety, narządy organizmu ludzkiego wzajemnie się przysłaniają, co prowadzi do nakładania się na siebie obrazów poszczególnych struktur wewnętrznych człowieka(przy przepuszczeniu jednej wiązki).
Zauważono, że wykonanie większej (niż jeden) liczby zdjęć radiologicznych z różnych pozycji lampy i detektora względem badanego obiektu, a następnie obejrzenie zdjęć w stroboskopie prowadzi do poprawienia jakości obrazu. Wprowadzono więc ruch lampy rentgenowskiej detektora względem obiektu.

\subsubsection{Generacje tomografów}
Od czasów wynalezienia tomografu zmieniały koncepcje i usprawnienia mające na celu zwiększenie ich szybkości i wydajności.
W związku z tym powastała poniższa klasyfikacja na generacje:
\begin{itemize}
\item Generacja I - skaner składał się z pojedynczego emitera i detektora. Lampa i detektor wykonywały ruchy translacyjne i rotacyjne.
\item Generacja II - zwiekszono liczbę detektorów co zmniejszyło liczbę ruchów translacyjnych lampy.
\item Generacja III - wyeliminowano ruch translacyjny poprzez rozmieszczenie detektorów na łuku pierścienia obracającego się razem z lampą dookoła pacjenta.
\item Generacja IV - detektory umieszczone zostały na stałe na pierścienu, ruch obrotowy wykonuje tylko lampa.
\end{itemize}

\section{Opis rozwiązania}
Poniżej znajduje się kilka pojęć, których zrozumienienie i wykorzystanie wymagane jest do rozwiązania projektu.
\subsubsection{Transformata Radona}
W roku 1905 W. Radon udowodnił następujące twierdzenie: „Obraz obiektu dwuwymiarowego można zrekonstruować na podstawie nieskończonej ilości rzutów jednowymiarowych”. Rzutowanie to odpowiada wykonywaniu na obiekcie pewnej transformacji, nazywanej Transformacją Radona.
Po dokonaniu transformacji z otrzymanych wyników otrzymyje się sinogram, czyli wykres będący wizualizacją owych wyników.
Dokonanie na sinogramie Odwrotnej Transformacji Radona umożliwia zrekonstruowanie obrazu obiektu. Zrekonstruowany obraz zawierał będzie pewne zniekształcenia, lecz wraz ze wzrostem liczby projekcji powinien co raz bardziej odzwierciedlać obraz sprzed transformacji.

\subsubsection{Algorytm Bresenham'a}
Służy do wyznaczania pikseli leżących na odcinku między dwoma punktami. W naszym zastosowaniu służy do wyznaczania pikseli obrazu, przez które przechodzi promień oraz wyznaczenia wartości pochłoniętego promieniowania.

\subsubsection{Technologia}
Program zaimplementowany zostanie w języku Python(wersja 2.7) z wykorzystaniem następujących bibliotek:
\begin{itemize}
\item skimage - wczytanie obrazu
\item numpy - działania na macierzach
\end{itemize}
\newpage
Przyjęto model tomografu o następujących parametrach:
\begin{itemize}
\item alfa - kąt o który obracany jest układ emiterów
\item l - odległość układu emiterów od macierzy obrazu
\item szerokość wiązki
\item szerokość filtra
\end{itemize}

\section{Plan dalszej pracy}
Przyjęto stworzenie symulatora tomografu czwartej generacji. Zaimplementowano już algorytm Bresenham'a, wyznaczający piksele w macierzy znajdujące się na drodze  "promienia". Wczytano obraz w skali szarości. Obecnie prace trwają nad sposobem implementacji transformaty Radona.
Do wykonania pozostało:
\begin{itemize}
\item implementacja transformaty Radona pozwalajaca na wygodny dobór parametrów
\item implementacja filtrów
\item budowa interfejsu do zmiany parametrów
\end{itemize}

\bibliographystyle{apalike}
\bibliography{sample}

\end{document}